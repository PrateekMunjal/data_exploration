%% Generated by Sphinx.
\def\sphinxdocclass{report}
\documentclass[letterpaper,10pt,english]{sphinxmanual}
\ifdefined\pdfpxdimen
   \let\sphinxpxdimen\pdfpxdimen\else\newdimen\sphinxpxdimen
\fi \sphinxpxdimen=.75bp\relax

\PassOptionsToPackage{warn}{textcomp}
\usepackage[utf8]{inputenc}
\ifdefined\DeclareUnicodeCharacter
% support both utf8 and utf8x syntaxes
\edef\sphinxdqmaybe{\ifdefined\DeclareUnicodeCharacterAsOptional\string"\fi}
  \DeclareUnicodeCharacter{\sphinxdqmaybe00A0}{\nobreakspace}
  \DeclareUnicodeCharacter{\sphinxdqmaybe2500}{\sphinxunichar{2500}}
  \DeclareUnicodeCharacter{\sphinxdqmaybe2502}{\sphinxunichar{2502}}
  \DeclareUnicodeCharacter{\sphinxdqmaybe2514}{\sphinxunichar{2514}}
  \DeclareUnicodeCharacter{\sphinxdqmaybe251C}{\sphinxunichar{251C}}
  \DeclareUnicodeCharacter{\sphinxdqmaybe2572}{\textbackslash}
\fi
\usepackage{cmap}
\usepackage[T1]{fontenc}
\usepackage{amsmath,amssymb,amstext}
\usepackage{babel}
\usepackage{times}
\usepackage[Bjarne]{fncychap}
\usepackage{sphinx}

\fvset{fontsize=\small}
\usepackage{geometry}

% Include hyperref last.
\usepackage{hyperref}
% Fix anchor placement for figures with captions.
\usepackage{hypcap}% it must be loaded after hyperref.
% Set up styles of URL: it should be placed after hyperref.
\urlstyle{same}
\addto\captionsenglish{\renewcommand{\contentsname}{Contents:}}

\addto\captionsenglish{\renewcommand{\figurename}{Fig.\@ }}
\makeatletter
\def\fnum@figure{\figurename\thefigure{}}
\makeatother
\addto\captionsenglish{\renewcommand{\tablename}{Table }}
\makeatletter
\def\fnum@table{\tablename\thetable{}}
\makeatother
\addto\captionsenglish{\renewcommand{\literalblockname}{Listing}}

\addto\captionsenglish{\renewcommand{\literalblockcontinuedname}{continued from previous page}}
\addto\captionsenglish{\renewcommand{\literalblockcontinuesname}{continues on next page}}
\addto\captionsenglish{\renewcommand{\sphinxnonalphabeticalgroupname}{Non-alphabetical}}
\addto\captionsenglish{\renewcommand{\sphinxsymbolsname}{Symbols}}
\addto\captionsenglish{\renewcommand{\sphinxnumbersname}{Numbers}}

\addto\extrasenglish{\def\pageautorefname{page}}

\setcounter{tocdepth}{1}



\title{DataExploration Documentation}
\date{Aug 15, 2020}
\release{0.0.1}
\author{Prateek Munjal}
\newcommand{\sphinxlogo}{\vbox{}}
\renewcommand{\releasename}{Release}
\makeindex
\begin{document}

\pagestyle{empty}
\sphinxmaketitle
\pagestyle{plain}
\sphinxtableofcontents
\pagestyle{normal}
\phantomsection\label{\detokenize{index::doc}}



\chapter{DataExploration utils}
\label{\detokenize{index:dataexploration-utils}}\phantomsection\label{\detokenize{index:module-utils}}\index{utils (module)@\spxentry{utils}\spxextra{module}}\index{collect\_hu\_values() (in module utils)@\spxentry{collect\_hu\_values()}\spxextra{in module utils}}

\begin{fulllineitems}
\phantomsection\label{\detokenize{index:utils.collect_hu_values}}\pysiglinewithargsret{\sphinxcode{\sphinxupquote{utils.}}\sphinxbfcode{\sphinxupquote{collect\_hu\_values}}}{\emph{data}}{}
Collect stats on Housenfeld units of the masks for scaling and visualizations.

INPUT:
mask\_paths: list, absolute paths of masks
data: object of class Data

OUTPUT:

\end{fulllineitems}

\index{dump\_data() (in module utils)@\spxentry{dump\_data()}\spxextra{in module utils}}

\begin{fulllineitems}
\phantomsection\label{\detokenize{index:utils.dump_data}}\pysiglinewithargsret{\sphinxcode{\sphinxupquote{utils.}}\sphinxbfcode{\sphinxupquote{dump\_data}}}{\emph{img\_size}, \emph{img\_res}, \emph{img\_area}, \emph{CT\_paths}, \emph{save\_fpath=None}}{}
Dumps the data in a csv file saved at path: save\_fapth.

INPUT:
save\_fpath: str, Path where we dump all the dataset statistics. Default: None

OUTPUT:
None

\end{fulllineitems}

\index{get\_dataset\_stats() (in module utils)@\spxentry{get\_dataset\_stats()}\spxextra{in module utils}}

\begin{fulllineitems}
\phantomsection\label{\detokenize{index:utils.get_dataset_stats}}\pysiglinewithargsret{\sphinxcode{\sphinxupquote{utils.}}\sphinxbfcode{\sphinxupquote{get\_dataset\_stats}}}{\emph{CT\_paths}, \emph{Mask\_paths}}{}
INPUT:
CT\_paths: list, A list containing absolute paths to CTs
Mask\_paths: list, A list containing absolute paths to Masks

OUTPUT:
imageSizes, imageResolutions, imageAreas

NOTE:
If save\_fpath is None, then we do not save the stats.

\end{fulllineitems}

\index{get\_nonzero\_voxel\_count() (in module utils)@\spxentry{get\_nonzero\_voxel\_count()}\spxextra{in module utils}}

\begin{fulllineitems}
\phantomsection\label{\detokenize{index:utils.get_nonzero_voxel_count}}\pysiglinewithargsret{\sphinxcode{\sphinxupquote{utils.}}\sphinxbfcode{\sphinxupquote{get\_nonzero\_voxel\_count}}}{\emph{slice\_count\_all}, \emph{thresh=0}}{}
get voxelcounts \textgreater{} thresh

\end{fulllineitems}

\index{get\_size\_resolution() (in module utils)@\spxentry{get\_size\_resolution()}\spxextra{in module utils}}

\begin{fulllineitems}
\phantomsection\label{\detokenize{index:utils.get_size_resolution}}\pysiglinewithargsret{\sphinxcode{\sphinxupquote{utils.}}\sphinxbfcode{\sphinxupquote{get\_size\_resolution}}}{\emph{sitk\_img\_object}}{}
Returns the size and voxel spacing of sitk\_image\_object

\end{fulllineitems}

\index{load\_image() (in module utils)@\spxentry{load\_image()}\spxextra{in module utils}}

\begin{fulllineitems}
\phantomsection\label{\detokenize{index:utils.load_image}}\pysiglinewithargsret{\sphinxcode{\sphinxupquote{utils.}}\sphinxbfcode{\sphinxupquote{load\_image}}}{\emph{img\_path}}{}
This function loads the image specified by img\_path.
\begin{quote}\begin{description}
\item[{Parameters}] \leavevmode
\sphinxstyleliteralstrong{\sphinxupquote{img\_path}} (\sphinxstyleliteralemphasis{\sphinxupquote{str}}) \textendash{} Absolute path to the image file

\item[{Returns}] \leavevmode
sitk object corresponding to image loaded

\item[{Return type}] \leavevmode
SimpleITK

\end{description}\end{quote}

\end{fulllineitems}

\index{pl\_get\_dataset\_stats() (in module utils)@\spxentry{pl\_get\_dataset\_stats()}\spxextra{in module utils}}

\begin{fulllineitems}
\phantomsection\label{\detokenize{index:utils.pl_get_dataset_stats}}\pysiglinewithargsret{\sphinxcode{\sphinxupquote{utils.}}\sphinxbfcode{\sphinxupquote{pl\_get\_dataset\_stats}}}{\emph{id}, \emph{data}}{}
Evaluate size, resolution and area of ct\_scan

\end{fulllineitems}

\index{plot\_hist() (in module utils)@\spxentry{plot\_hist()}\spxextra{in module utils}}

\begin{fulllineitems}
\phantomsection\label{\detokenize{index:utils.plot_hist}}\pysiglinewithargsret{\sphinxcode{\sphinxupquote{utils.}}\sphinxbfcode{\sphinxupquote{plot\_hist}}}{\emph{array}, \emph{title}, \emph{x\_label}, \emph{y\_label}, \emph{n\_bins=200}}{}
Plots the histogram where values are specified in array and groups
the data using n\_bins.

INPUT:
array: np.ndarray, The array values.
title: str, the title of plot
x\_label: str, label shown on x-axis
y\_label: str, label shown on y-axis
n\_bins: int, the number of bins

\end{fulllineitems}



\chapter{Indices and tables}
\label{\detokenize{index:indices-and-tables}}\begin{itemize}
\item {} 
\DUrole{xref,std,std-ref}{genindex}

\item {} 
\DUrole{xref,std,std-ref}{modindex}

\item {} 
\DUrole{xref,std,std-ref}{search}

\end{itemize}


\renewcommand{\indexname}{Python Module Index}
\begin{sphinxtheindex}
\let\bigletter\sphinxstyleindexlettergroup
\bigletter{u}
\item\relax\sphinxstyleindexentry{utils}\sphinxstyleindexpageref{index:\detokenize{module-utils}}
\end{sphinxtheindex}

\renewcommand{\indexname}{Index}
\printindex
\end{document}